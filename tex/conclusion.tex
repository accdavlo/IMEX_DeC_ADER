%In our work, we have analysis the implicit and implicit-explicit ADER and DeC methods concerning their stability properties. 
%Therefore, we have first reformulate them as RK methods and have them investigated in respect on the chosen order, method and quadrature nodes. 
%Different from our previous work \cite{Han_Veiga_2021} where the explicit versions have been investigated, we have derived significant variations in stability behaviour, ranging from A-stable to bounded stability regions. In summary, the implicit (and implicit-explicit) ADER methology demonstrate a more stable behaviour compare to the DeC framework in general. 
%After the ODE case, we have further extended our analysis to the PDE case focusing on advection-diffusion and advection-dispersion equations inspired by the works \cite{TanChenShu_ImEx_Stability, WangShuZhang_LDG1_2015}. For the space discretization, we employed up-to-date finite difference stencils and derived CFL-like stability condition via a  von Neumann stability analysis. In particular, we expand the investigation of \cite{TanChenShu_ImEx_Stability}. By introducing two new auxillary coefficients  for the advection-diffusion equation, we obtain equivalent conditions to the ones in \cite{TanChenShu_ImEx_Stability} but the new ones  are as well also  independent on spatial-constraints.
%We establish precise boundaries for relevant coefficients for advection-diffusion and advection-dispersion in and provide suggestions regarding the suitability of specific schemes. \\
%In the future, further research directions would be to change the space-discretization and focus on continuous and discontinuous Galerkin formulations instead \cite{ortleb2023stability, zbMATH07137361}. Additionally, the investigation of stability in context of nonlinear problems would also be desirable. Here, one may focus on the entropy production of such schemes following the works  \cite{zbMATH07086321, zbMATH06928679, oeffner2020}.

In our study, we have analyzed the implicit and implicit-explicit ADER and DeC methods in terms of their stability properties. To this end, we initially reformulated them as RK methods and investigated them based on the selected order, method, and quadrature nodes. Unlike our prior work \cite{Han_Veiga_2021}, which focused on explicit versions, we observed significant variations in stability behavior, ranging from A-stable to bounded stability regions. In general, the implicit (and implicit-explicit) ADER methodology demonstrated greater stability compared to the DeC framework.

Subsequently, after examining the ODE case, we extended our analysis to the PDE case, concentrating on advection-diffusion and advection-dispersion equations inspired by previous works \cite{TanChenShu_ImEx_Stability, WangShuZhang_LDG1_2015}. For space discretization, we utilized modern finite difference stencils and derived CFL-like stability conditions through von Neumann stability analysis. 
Notably, we expanded upon the investigation of \cite{TanChenShu_ImEx_Stability} by introducing two new auxiliary coefficients for the advection-diffusion equation. 
These coefficients yielded equivalent conditions to those in \cite{TanChenShu_ImEx_Stability}, but they do not depend on the spatial discretization. 
We established precise boundaries for relevant coefficients for advection-diffusion and advection-dispersion and offered recommendations regarding the suitability of specific schemes.

Looking ahead, potential research directions include exploring different space discretization methods and focusing on continuous and discontinuous Galerkin formulations \cite{ortleb2023stability, zbMATH07137361}. Additionally, investigating stability in the context of nonlinear problems would be desirable, particularly focusing on the entropy production of such schemes as suggested by previous works \cite{zbMATH07086321, zbMATH06928679, oeffner2020}, or
using the add-and-subtract version of the implicit ADER and DeC to study the stability in the same style of \cite{tan2022stability}.